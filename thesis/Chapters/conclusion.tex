% !TeX spellcheck = en_US
\chapter{Conclusion and future work} % Main chapter title

\label{chap:conclusion} % Change X to a consecutive number; for referencing this chapter elsewhere, use \ref{ChapterX}

\lhead{Chapter \ref*{chap:conclusion}. \emph{Conclusion and future work}} % Change X to a consecutive number; this is for the header on each page - perhaps a shortened title


\section{Lessons learned}
% man kann sich hier auch wiederholen; für leute, damit sie nicht nochmal die gleichen fehler machen-

\section{Conclusion}
The thesis proofed that implementing a cluster with \ac{Windows} nodes is not easy to handle. The switch from Docker to ContainerD is still an ongoing progress and is not completed yet.
The Hyper-V isolation as \ac{CRI} abstracts some issues with the underlying container isolation. Problems on the operating system level are mostly solved and inputs are properly validated so that errors are more understanding.
The host-process isolation offered by \ac{Windows}, however, is providing a pure abstraction layer to the containers.
The results showed many pitfalls that one has to be aware of. This includes missing quality in documentation and difficulties in the error handling of the \ac{K8s} systems. When it comes to the community, hard to find help especially for the cases where ContainerD is in use.
After containerization of the cluster

% TODO: Proof bringen, dass community klein ist und viele tickets wegen triage geschlossen werden. und unbeantwortet bleiben.

\section{Future work}
The results showed that there are still some points open. Moreover there are plenty of opportunities using the findings as a basis and implement a full functional cluster on top of it. This section mentions what can be optimized in the future and provides an outcast for future implementations.


- Images verkleinern - nur die Dateien ins Image bundlen, die für den entsprechenden Service notwendig sind.

- Linux  port and cluster based on linux
- Automated image building using Packer.io (for multiple platforms)
- Ranger for streamlining kubernetes deployment
- Put certificate files into container via file mount - Design?

- Falls OT sowieso in Container gebaut werden muss, dann kann dieser Schritt auch automatisiert werden (clone aus git, bauen mit Build Tools)
- Dann ist auch nicht mehr nötig, Deployment Verzeichnis manuell auf jeden Node zu kopieren.


- Blockzitat: K8s checkliste, zur korrekten Funktionsweise des Netwerks -> Funktioniert nicht

- GPU einbinden für compute services?

