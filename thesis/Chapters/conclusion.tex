% !TeX spellcheck = en_US
\chapter{Conclusion and future work} % Main chapter title

\label{chap:conclusion} % Change X to a consecutive number; for referencing this chapter elsewhere, use \ref{ChapterX}

\lhead{Chapter \ref*{chap:conclusion}. \emph{Conclusion and future work}} % Change X to a consecutive number; this is for the header on each page - perhaps a shortened title


\section{Lessons learned}
% man kann sich hier auch wiederholen; für leute, damit sie nicht nochmal die gleichen fehler machen-

\section{Conclusion}

\section{Future work}
- Linux  port and cluster based on linux
- Automated image building using Packer.io (for multiple platforms)
- Ranger for streamlining kubernetes deployment
- Images verkleinern - nur die Dateien ins Image bundlen, die für den entsprechenden Service notwendig sind.
- Put certificate files into container via file mount - Design?

- Falls OT sowieso in Container gebaut werden muss, dann kann dieser Schritt auch automatisiert werden (clone aus git, bauen mit Build Tools)
- Dann ist auch nicht mehr nötig, Deployment Verzeichnis manuell auf jeden Node zu kopieren.


- Blockzitat: K8s checkliste, zur korrekten Funktionsweise des Netwerks -> Funktioniert nicht

- GPU einbinden für compute services?

