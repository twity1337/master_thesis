% !TeX spellcheck = en_US

% ==== Containerized multi-level deployment for a distributed adaptive microservice application ====

% Chapter 1

\chapter{Introduction} % Main chapter title

\label{chap:introd} % For referencing the chapter elsewhere, use \ref{Chapter1} 

\lhead{Chapter  \ref*{chap:introd}. \emph{Introduction}} % This is for the header on each page - perhaps a shortened title


The containerization of applications became more and more popular in the recent decade. Running applications in an isolated environment makes them more easy to manage and easier to deploy. The lightweight method of isolating processes is the basis for deploying applications on a cluster environment.
Deployed applications on distributed systems can nowadays profit from more reliability and availability. Additionally, the services can gain a higher performance due to horizontal scaling mechanisms. This means, instead of increasing performance by upgrading the hardware, the services are accessible on multiple servers.
By having a microservice architecture, complex systems are well prepared for a distributed system. The several services of the application are responsible for just their own task and thus can be scaled horizontally.

This thesis explores the feasibility of deploying a \ac{Windows} based microservice application on a cluster and gives an overview about the faced challenges. The feasibility is showcased on the basis of \ac{OT}. \ac{OT} is a distributed application for computer aided design and physics simulation, where the computations can be done server-side, while the operation by the user stays client-side.

The majority of deployments is still performed on Linux, even though deploying applications for \ac{Windows} is gaining more and more demand. Recently Microsoft introduced with \enquote{Windows Containers} its own container technology\cite{Microsoft.2022} and \ac{K8s} announced the drop of Docker support\cite{Kubernetes.2020}. Subsequently, the container technology on \ac{Windows} shifted towards Windows Containers.
Since \ac{OT} is based on \ac{Windows}, the deployment has to be achieved there as well. As a consequence, the cluster is added 


\begin{comment}
Nevertheless, the benefits come with the price of required application changes and higher system requirements.

Because 
Microservice applications consist of multiple services where each service fulfills its specific task. They meet the requirements for 


Applications that are served on a cluster need to fulfill the corresponding architecture requirements for a cluster environment.  and they 

This thesis explores the design and development of an cluster for an microservice application.
With regards to the automation of steps.

The requirements for serving applications in a cluster are the underlying

 the dependency on external services is high. 

The application under study, \acf{OT}, is well prepared for a cluster deployment.

\ac{K8s} was mainly used on Linux systems in the past

 there is a growing need to deploy microservice applications on Windows platforms as well. 

The thesis examines the challenges involved in deploying such an application on Windows


The research field of cluster deployment with ContainerD is still in the beginning state for \ac{Windows}.
\end{comment}

Through this work, the thesis aims to provide a valuable resource for research groups who want to deploy applications on Windows using common cluster management tools.

% TODO: Wenig vorarbeiten
% TODO: Multi-layer to explain

\section{Scope}
\section{Intended audience}


\section{Outline}
The first chapter (\autoref{chap:background}) of the thesis provides a brief introduction into \ac{OT} and its architecture. Furthermore, it introduces its encryption protocol and gives an outline about the thesis. In \autoref{chap:design} the major design decisions are explained. It dives into the used container engine and the chosen cluster management software and their alternatives.
The design decisions are implemented in \autoref{chap:implementation}. This is where the application is adapted by the required changes. The application is containerized, and the cluster is set up. Moreover, the probed options for designing the cluster are presented and the deployment of an application is shown.
The findings are discussed in \autoref{chap:results}. A deeper look is taken in challenges that occurred during the implementation and the underlying reasons.
The thesis ends with \autoref{chap:conclusion} and gives an overview about the learned lessons and the conclusion. Additionally, it shows points of contact for future work.
