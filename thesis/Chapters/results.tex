% !TeX spellcheck = en_US
\chapter{Results} % Main chapter title

\label{chap:results} % Change X to a consecutive number; for referencing this chapter elsewhere, use \ref{ChapterX}

\lhead{Chapter \ref*{chap:results}. \emph{Results}} % Change X to a consecutive number; this is for the header on each page - perhaps a shortened title

The development of the \ac{K8s} cluster and the containerization of \ac{OT} hides several pitfalls. The results are discussed in this chapter.

\section{Containerization}
Issues occurred during the containerization of the application. This involves pitfalls with the development of the container images and the special cases with the new \ac{Windows} host-process container technology. The required considerations are discussed in this section.

\subsection{Container manifest}\label{chap:results.container_manifest}
Even though the format of the container manifests "Containerfile" is compatible to the proprietary "Dockerfile" format from Docker, the \acp{CRI} do not follow the specification everywhere\cite. This was an issue while writing a Containerfile for the ContainerD \ac{CRI}. Especially in cases where line breaks in the Containerfile were necessary to shorten long lines and increase readability. ContainerD is treating line breaks paths in string notation different compared to paths in JSON array notation and is not following the specification\cite{ManKier.20230222}.
\begin{lstlisting}[caption={Containerfile entrypoint specification across multiple lines in text format.}, label=lst:results.containerfile-entrypoint.text]
ENTRYPOINT open_twin.exe \
Service.dll
\end{lstlisting}
\autoref{lst:results.containerfile-entrypoint.text} shows an example of the problem. While the entry point in Docker is interpreted as \tcode{open\_twin.exe Service.dll}, the interpreter in ContainerD only reads the first line as entry point and ignores the line break character "\textbackslash". Therefore it results in just \tcode{open\_twin.exe} as interpreted entry point. This causes troubles during build and execution of a container image. The container image is built in Docker, whereas it does not run in ContainerD. Since, ContainerD on \ac{Windows} cannot build container images, it does also not work the other way around. To overcome this flaw, the \textit{ENTRYPOINT} definition has to be written in JSON notation. If defined as in \autoref{lst:results.containerfile-entrypoint.json} the interpreter 
\begin{lstlisting}[caption={Containerfile entrypoint specification across multiple lines in JSON format.}, label=lst:results.containerfile-entrypoint.json]
ENTRYPOINT ["open_twin.exe", 
"Service.dll"]
\end{lstlisting}


\subsection{Windows Base Image}
Because of the light-weight isolation layer of containers, the \ac{OS} of container images must match those on the host machine. Compared to virtualization this light-weight approach reduces the isolation on the one hand, but improves performance on the other.
It is a requirement to have the same \ac{OS} kernel when running container images on Linux. However, on \ac{Windows} host-process isolation it is a requirement to have the exact same build version for the base image as for the host machine. This means, a container based on Windows Server 2019 (Version 10.0.17763.4010) cannot run on Windows Server 2022 (Version 10.0.20348.1547).
The isolation that is built on Hyper-V virtualization does not have this requirement. 
Microsoft provides several image tags (\enquote{ltsc2019}, \enquote{ltsc2022}, and so on) for different kinds of \ac{OS} versions\cite{MattbriggsMicrosoft.20230214,Microsoft.20230313}. They also offer one general image tag for multiple architectures\cite{Microsoft.20230313}. However, they do not offer a more generous tag having a broad version coverage, even though the \acp{CRI} support it. This is why the container image tag always has to match the host \ac{OS}.

There are two different ways to solve this flaw of \ac{OS} inconsistency. One method is given by \ac{K8s}. \ac{K8s} supports filtering based on the build number of a node\cite{Brasmithms.20230313}. For this, a separate \tcode{nodeSelector} key is defined as shown in \autoref{lst:results.windows_build}.
\begin{lstlisting}[caption={\ac{K8s} manifest with node selector for specific Windows build.}, label=lst:results.windows_build, language=yaml]
nodeSelector:
  "beta.kubernetes.io/os": windows
  "beta.kubernetes.io/osbuild": "14393.1715"
\end{lstlisting}
A second method is to build different versions of one image and assign them the same tag. If rolled out via an image registry, the node would automatically retrieve the suitable architecture. Google showed that so called \enquote{multi-arch} images can also be created with multiple build versions\cite{Google.20230313}.


\subsection{Local images namespace}
If images are locally imported, they have to be in the ContainerD namespace "k8s.io" to get recognized by \ac{K8s}. However, ContainerD on \ac{Windows} is not able to build images because the \ac{Windows} support for the build extension \enquote{build-kit} is still under development\cite{Microsoft.20221225}. If images are built with Docker, they are not recognized by ContainerD and additionally get the default namespace assigned. Although, images pulled with \ac{K8s} from a image registry are getting imported correctly. The faulty namespace behavior cannot be changed during the image build, because the \ac{CLI} of Docker does not provide the necessary option. Thus the change of the namespace is required for locally imported images on the one hand, and it requires a transfer of the image from the Docker environment to the ContainerD environment on the other hand.
So the only way to use local images without uploading them to an image registry is to perform the following steps for each image:
\begin{enumerate}
	\item Build the image in Docker (using \texttt{docker build})
	\item Export the image to an archive file (using \texttt{docker save})
	\item Load the image to ContainerD to correct namespace (using \texttt{ctr --namespace k8s.io image import})
\end{enumerate}
Because the application images of \ac{OT} become very large (\~{}2.8 Gigabyte), the three steps taking about five to ten minutes each. Depending on the amount of container images to process and the performance of the host computer, this consumes about half an hour.

To automate this process for a batch of container files, the PowerShell script \textit{build-image.ps1} was created. The script reads container files in a directory and afterwards performs the described steps on it as shown in \autoref{lst:results.image_namespace_build}.
The defined variable \tcode{\$ContainerRegistry} is set to the address of the image registry for tagging the image. In this case it is the stub domain \enquote{local.dev}. If an image is not tagged with a proper domain name, the \ac{CRI} is unable pull it and it cannot be found by \ac{K8s}.
\tcode{\$BaseName} is the filename of the container file and therefore the name of the image. The created temporary file \textit{container.tar} is deleted by the script after it has finished.

\begin{lstlisting}[caption={Crucial commands used in the custom PowerShell script to automate process of image building.}, label=lst:results.image_namespace_build, language=powershell, morekeywords={docker, ctr}]
$containerfiles | ForEach-Object {
  docker build -f "$($_)" -t "$($ContainerRegistry)/opentwin-$($_.BaseName):latest" \<*@ \Suppressnumber @*>
    "..\..\Deployment\"<*@ \Reactivatenumber @*>
  docker save "$($ContainerRegistry)/opentwin-$($_.BaseName):latest" -o container.tar
  ctr --namespace k8s.io image import .\container.tar }
\end{lstlisting}


\section{Kubernetes}
Flaws with the documentation and the compatibility between different versions have been found while working with \ac{K8s}. This highlights the special treatment that is required to add \ac{Windows} nodes to the cluster. The details are described in this section.
 
\subsection{System requirements}
Because \ac{K8s} does not support containers based on Hyper-V isolation\cite{Kubernetes.20230227}, host-process containers must be used.
For using host-process containers on \ac{Windows}, at least Windows Server in version 2019 with build number \enquote{17763.379} is required or the update \enquote{KB4489899} needs to be installed\cite{GitHubKubernetesSIGWindowsTools.20230213}. Furthermore, \ac{K8s} also has the fixed requirement of the usage of \enquote{Windows Server 2019} and above\cite{Kubernetes.20230227}. This reduces the subset of compatible node \ac{OS}, because \ac{Windows} client machines are not able to host containers for the usage in \ac{K8s} and a minimum version of \enquote{Windows Server} is required.
Moreover, the update cannot be installed manually which means it is impossible to get support for \enquote{Windows 10} based systems that are not part of this group. In the business area, Microsoft sells major upgrades of Windows Server as separate product. That is why the necessity of upgrading to at least \enquote{Windows Server Build 17763} can induce a higher expense.

Additionally, there is the incompatibility with the control plane on \ac{Windows}. This requires an additional Linux node for running the \ac{K8s} control plane and hence is not conducive for a homogeneous system landscape. 

\subsection{Documentation}
The official documentation of \ac{K8s} also hides some obstacles while implementing a cluster. It turned out to be incomplete or outdated on some pages.
To be more precise, there is partially work in progress and while consuming the documentation it is hard to detect if a specific article is applicable for the current version of \ac{K8s}.

For example, this became fundamental on the page \enquote{Adding Windows nodes}\cite{Kubernetes.20220419}. At the current time being, the actual page is only available as an archived version for \ac{K8s} version \enquote{v1.23}. However, if the same \ac{URL} is called for the unarchived version, the user gets redirected to a similar page \enquote{Creating a cluster with kubeadm}\cite{Kubernetes.20221107}. This can easily happen if a link is followed from an issue or a web search. Reading the page, people could assume that they can follow the steps described there to add a \ac{Windows} node to a cluster. However, the fact that the guide is aligned to Linux is only barely mentioned and thus is misleading.
That also means that there is no guide for adding \ac{Windows} nodes in the current version of the documentation of \ac{K8s}.

The guide for adding \ac{Windows} nodes provided from \ac{SIG} \enquote{Windows Tools}\cite{GitHubKubernetesSIGWindowsTools.20230213} has a more streamlined approach and benefits from provided PowerShell scripts, like \enquote{PrepareNode.ps1} (which is used in \autoref{chap:implementation.setup_worker}). However, due to the small community, the guide still stays faulty or outdated. For example, the guide shows a warning, \enquote{The instructions and scripts in the directory DO NOT configure a CNI solution for Windows nodes running containerd. There is a work-in-progress PR to assist in this at \#239}\cite{GitHubKubernetesSIGWindowsTools.20230213}. But the guide does explain how to configure a \ac{CNI} and the linked pull request is merged since December 2022. Thus, the warning is misleading.
Furthermore, it reads \enquote{Do not forget to add --cri-socket [\dots] at the end of the join command, if you use ContainerD}\cite{GitHubKubernetesSIGWindowsTools.20230213}. The explicit appending of the \ac{CRI} socket however is not necessary for versions higher than \enquote{v1.25} as mentioned in \autoref{chap:implementation.setup_worker}.

The guide from \ac{SIG} \enquote{Windows Tools} was still work in progress while the cluster was set up. A change in the documentation for applying a required daemon set on the cluster was only added after three months. Taking into account that Windows nodes are supported since over year, a functional documentation was expected.

\section{Interfaces within the abstraction layers}
The communication within the different abstraction layers is a crucial part of a complex system like \ac{K8s}. Kubectl has to communicate the commands to the container runtime and Flannel, which communicate with the \ac{OS} layer, and the hosted applications exchange their status with the \ac{CRI}. 
It is crucial for the correctness of the system to have all parts compatible to each other. The flaws related to these problems are described here. 

\subsection{Version interoperability}
The huge amount of systems and third-party tools the whole \ac{K8s} landscape consists of burdens the risk of version incompatibilities. 
% This was identified while setting up the cluster and the \ac{CNI}.
The versions of Flannel and the \ac{CNI} that are written in the manifest have to match the version installed on every node in the cluster (including the control plane). For example, the Flannel version on the node system has to match the version defined in the Flannel pod manifest. This induces the necessity of string replacements to hand in the correct versions and increases complexity.

Another finding covers the fact that support for Docker was removed from \ac{K8s} before properly introducing the alternative ContainerD to not only Linux but also \ac{Windows}. The majority of guides that can be found online still stick to Docker as basis. Guides that cover specific cases with ContainerD are rare.
This can also be seen on the missing build support for \ac{Windows} containers in ContainerD. Until now the container images have to be built in Docker and need to be shifted to ContainerD, because the required tools (\enquote{build-kit}) are not yet available for \ac{Windows}\cite{Microsoft.20221225}.

\subsection{Error handling}
The additional abstraction layer for the \ac{HCS} is loosing information in error cases. Errors on the \ac{OS} level are passed to the user, but \ac{Windows} does not provide much additional information alongside the error. For example, in cases with an incorrect container image entry point as described in \autoref{chap:results.container_manifest} the error message returned to the \ac{K8s} control plane is \enquote{hcs::CreateComputeSystem session: The system cannot find the file specified.: unknown}. With no further arguments given, it is hard to break the issue down to the actual file.

Furthermore, error cases on one \ac{Windows} version differ from the one on other versions. This has been found while setting up the \enquote{kube-proxy} pod with the exact same steps on Windows Server 2022 and Windows Server 2019. While the error message on 2019 is rather generic (\enquote{The directory name is invalid.}), the error message for 2022 is more explanatory (\enquote{IP address is either invalid or not part of any configured subnet(s)}).

% - Cluster networking throws weird error messages when performed inside hyper-v vm ("Directory not found")
Additionally, the error reporting of \ac{OT} itself can still be more improved. The logging mechanisms on \ac{OT} are still under development and thus logging is rare. If a service crashes it is tricky to reproduce the cause. Especially for the compute services the generated output does not contain useful information. Due to the fact that the console window stays not open after a crash, the output is hard to catch.
In some cases throughout the code base exceptions are not caught or just swallowed, where it is better to treat them and return the exception message.
Also, since Rust cannot catch foreign exceptions, they cannot be exchanged between the service \ac{DLL} and the main executable. This is why all exceptions have to be fully treated inside the service \ac{DLL}. 


\subsection{Database server}

- TODO: MongoDB server container image cannot be initialized easily on Windows



